% Created 2018-06-21 Thu 12:30
\documentclass[8pt]{beamer}
\usepackage[sc,osf]{mathpazo}   % With old-style figures and real smallcaps.
\linespread{1.025}              % Palatino leads a little more leading
% Euler for math and numbers
\usepackage[euler-digits,small]{eulervm}
%\documentclass[10pt]{llncs}
%\usepackage{llncsdoc}
\usepackage{minted}
\usepackage[utf8]{inputenc}
\usepackage[T1]{fontenc}
\usepackage{fixltx2e}
\usepackage{graphicx}
\usepackage{longtable}
\usepackage{float}
\usepackage{wrapfig}
\usepackage{rotating}
\usepackage[normalem]{ulem}
\usepackage{amsmath}
\usepackage{textcomp}
\usepackage{marvosym}
\usepackage{wasysym}
\usepackage{amssymb}
\usepackage{hyperref}
\usepackage{polynom}
\renewcommand{\mod}[1]{\left( \texttt{mod}~#1 \right)}
\newcommand{\N}{\mathbb N}
\newcommand{\Z}{\mathbb Z}
\newcommand{\Q}{\mathbb Q}
\newcommand{\C}{\mathbb C}
\newcommand{\degree}{\texttt{degree}}
\tolerance=1000
%\usetheme{Antibes}
\addtobeamertemplate{navigation symbols}{}{%
    \usebeamerfont{footline}%
    \usebeamercolor[fg]{footline}%
    \hspace{1em}%
    \insertframenumber/\inserttotalframenumber
}


\author{Siddharth Bhat}
\date{Monday, Jan 18 2021}
\title{egg: Fast and extensible equality saturation}
\hypersetup{
  pdfkeywords={},
  pdfsubject={},
  pdfcreator={Emacs 24.5.1 (Org mode 8.2.10)}}
\begin{document}

\maketitle

\begin{frame}{egg: Fast and extensible equality saturation}
\begin{itemize}
\item What is eqsat?
\end{itemize}
\end{frame}

\begin{frame}{E-graphs}
\begin{itemize}
\item Developed for automated theorem provers.
\item Extend union-find to represent equivalence class of \emph{expressions}.
\item Closed under congruence.
\end{itemize}
\end{frame}

\begin{frame{E-graphs \emph{for equality saturation}}
\begin{itemize}
\item Read Phase: Find all matching rewrites
\item Write Phase: Perform rewrites
\item Canonicalize Phase: Merge equivalent nodes.
\end{itemize}
\end{frame}

\begin{itemize}
\item Abstract interpretation for each equivalence class
\end{itemize}

\begin{itemize}
\item Traditionally: E-graph always maintain congruence invariant
\item Egg: Maintain invariant in a separate pass.
\end{itemize}
\end{document}
